\documentclass{article} 

\title{URL-Shortener}
\author{Francesco Balzano}


\begin{document}
  \pagenumbering{gobble}
  \maketitle
  \newpage
  \tableofcontents
  \newpage
  \pagenumbering{arabic}
  
\section{Overview}  
URL shortening is the translation of a long Uniform Resource Locator (URL) into an abbreviated alternative that redirects to the longer URL.  A shortened URL may be desired for messaging technologies that limit the number of characters in a message, for reducing the amount of typing required if the reader is copying a URL from a print source, for making it easier for a person to remember, or for the intention of a permalink.\\  
The aim of this project is to provide the implementation of a distributed url-shortener service.


\section{Design choices}
There are three fundamental metrics to assess a distributed system: Consistency, Availability and Partition Tolerance. It is well known, after the publication of the CAP theorem, that it is impossibile to achieve all of them, at the same time, in a distributed system. \\ 
In this project, I have chosen to focus on availability and partition tolerance, at the expense of consistency. In particular, the consistency model is not a strong one.
In the following subsections, I list and explain the design choices that I made.


\subsection{}

\subsection{}

\subsection{}

\subsection{}

\subsection{}

\subsection{}

\subsection{}







\end{document}
