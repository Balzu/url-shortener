\documentclass{article} 

\title{URL-Shortener}
\author{Francesco Balzano}


\begin{document}
  \pagenumbering{gobble}
  \maketitle
  \newpage
  \tableofcontents
  \newpage
  \pagenumbering{arabic}
  
\section{Overview}  
URL shortening is the translation of a long Uniform Resource Locator (URL) into an abbreviated alternative that redirects to the longer URL.  A shortened URL may be desired for messaging technologies that limit the number of characters in a message, for reducing the amount of typing required if the reader is copying a URL from a print source, for making it easier for a person to remember, or for the intention of a permalink.\\  
The aim of this project is to provide the implementation of a distributed url-shortener service.


\section{Design choices}
There are three fundamental metrics to assess a distributed system: Consistency, Availability and Partition Tolerance. It is well known, after the publication of the CAP theorem, that it is impossibile to achieve all of them, at the same time, in a distributed system. \\ 
In this project, I have chosen to focus on availability and partition tolerance, at the expense of consistency. In particular, the consistency model is not a strong one.
In the following subsections, I list and explain the design choices that I made.


\subsection{API}

\paragraph{get}
Given the shorten url returns the original url, if present. \\
\texttt{get(shortUrl) -$>$ longUrl}

\paragraph{put}
Generates and returns the shortened url associated with the provided url. \\
\texttt{put(longUrl) -$>$  shortUrl}

\paragraph{remove}
Removes the couple \texttt{<shortened url, original url>}.\\
\texttt{remove(shortUrl)}


\subsection{Passive Replication}
Clients can communicate with every node. In case of WRITE operations, the node forwards the request to the primary, which executes it and sends back the response to the sending node, which in turn hands it back to the client. \\
In case of READ operation, the node that receives the request firstly checks its local store: if possible it directly answers the client (running the risk of providing inconsistent information, but answering more quickly), otherwise it forwards the READ request to the primary. \\
I have chosen the Passive Replication strategy because I think it should keep lower the number of version conflicts.

\subsection{Data Partitioning}
The partitioning strategy to map objects into nodes is a dynamic one: namely, it is employed Consistent Hashing. In case of node leave, due for instance to node crash or network partition, the keys asssigned to this node are automatically assigned to another, working node. Since we have replication of data and we use the gossip protocol to detect dead nodes, we are capable to face the leave of one or more nodes without compromising the functioning of the whole system. In other words, availability and partition tolerance are garanteed.

\subsection{Data Replication}
We want to achieve availability, so an asynchronous replication strategy is adopted. Namely, the primary node immediately answers to the client after an operation is performed. Messages to the backup nodes are sent periodically, and not after every update of the primary's data. This strategy reduces the latency of the communication client-primary, again at the expense of consistency. \\
Each primary node is associated 2 backup nodes, which are the next nodes clockwise in the ring. Every time a primary wishes to update its replicas, it sends data to these 2 nodes.

\subsection{Primary Failure}
The use of Consistent Hashing and the the specific data replication strategy adopted (backups are the next nodes clockwise in the ring) allows to avoid the need for a specific failover procedure. The failover procedure is instead automatic: if node \textit{i} is down, the space of keys that belonged to node \textit{i} will be automatically mapped to node \textit{(i+1) mod n}. Since node \textit{(i+1) mod n} is the backup of node \textit{i} it will have a copy of the keys of node \textit{i} (although possibly not updated), so the correct behaviour will be maintained also in case of failure of the primary without needing an explicit procedure.    


\subsection{}

\subsection{}

\subsection{}







\end{document}
